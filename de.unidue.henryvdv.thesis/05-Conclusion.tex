\chapter{Conclusion}
\label{sec:Conclusion}
In this thesis, three different implementations of machine-learning based classifiers using Support Vector Machines (SVM) for pronominal anaphora resolution were presented in order to determine the influence of corpus mined gender frequencies. This section will first of all summarize the implemented systems, followed by a discussion of the results and explanatory approaches. Finally, the last section will describe what could be done in future work and how the system could exert influence on other domains and scopes.

\section{Summary}
This work presented a pronominal anaphora resolution system using various information sources including syntactic, semantic, and corpus mined gender knowledge. A preprocessing pipeline including part-of-speech tagging, named entity recognition, constituency parsing, and dependency parsing was implemented in order to gain relevant information for feature assignment. In total, a feature set of 33 features was used (excluding gender probability features). Various implementations were done. First of all, a baseline system that assigns always the most recent noun phrase as antecedent was implemented. The first SVM classifier without usage of corpus mined gender information outperformed the baseline approach by far, indicating a clear benefit of machine learning in anaphora resolution. 

A first improvement consisted in using corpus mined gender probabilities for the antecedent candidates. Therefore, eight new features were added. The gender information was expressed through the mean and standard deviation of the \textit{Beta} distribution for each gender. The performance increased slightly. 

Another classifier was build using the same gender corpus frequencies, but assigning only the most frequent gender outcome as the correct gender for each considered noun. Four new features were added, one for each gender. This approach outperformed both previous implementations marginal.

Conspicious thereby is that both implementations using corpus mined gender information impaired its precision but increased its recall in comparison to the initial classifier. 

Corpus mined gender information seems to improve anaphora resolution on Wikipedia articles, yet the gain of 10 \% in performance \cite{bergsma2005automatic} reported through corpus mined frequencies could not be replicated. Additionally, the hard constraint gender implementation seems to have a higher impact on the classifier than the frequencies. Therefore, the hypothesis presented in the introduction (Section \ref{introductionMotivation}) can only be partially confirmed. 

\section{Discussion}
The reported results depend on many factors. First of all, Wikipedia articles are a very specific domain. The article guidelines described in Section \ref{wikicorefSec} aim for a high structure, implying as less ambiguity as possible. Wikipedia articles often focus on a single entity, which is mostly referred through pronouns. Therefore, cases with gender ambiguity in which the corpus mined frequencies give useful hints might be sparse. The error analysis indicated that a gender mismatch is accountable for approximately 16 \% of all errors. \cite{kennedy1996anaphora} associated 35 \% of their pronoun resolution errors on various texts (including inter alia news stories and magazine articles) to a mismatching gender. Since this work used different approaches and measures, a comparison is limited in its validity. Still, it might give a hint that gender mismatch is a lesser source of error in Wikipedia (for instance, it might be more likely to find those occurrences in dialogue).
Secondly, the implementation strategy differs from \cite{bergsma2005automatic} since no web mining frequencies were used. Although \cite{Bergsma:06} indicated that the frequency distribution used in this approach performed only slightly worse than their previous corpus- and web mined frequencies \citep{bergsma2005automatic}, it might influence the performance of the implemented classifier.\\
Additionally, \cite{bergsma2005automatic} split the corpus and web mined gender frequencies respectively in five categories. Therefore, the classifier had also the possibility to learn whether a word occurs with a reflexive, possessive, nominative, predicate, or designator (a further description is located in Section  \ref{section:bergsma2005automatic}). Not only the gender, but also in which case a noun occurs with a specific gender might have an impact on pronoun resolution.

The quality of preprocessing also affects the results, as parsing errors or falsely linked antecedents are unavoidable on huge corpora. In pronoun resolution some ambiguity might result of cases in which no clearly identifiable antecedent exists or multiple candidates are equally likely. 

Another impact on performance measures are cases in which world knowledge is required. For finding all antecedents, the algorithm needs to draw conclusions and recognize contextual relationships. \cite{baldwin1997cogniac} considered anaphora resolution as an ``A.I. complete'' task, meaning that the creation of a flawless anaphora resolution system requires a generally intelligent computer \citep{shapiro1992encyclopedia}. The Winograd Schema Challenge\footnote{A detailed description is available at http://www.nuance.com/company/news-room/press-releases/Winograd-Schema-Challenge.docx} is a test of artificial intelligence which adresses those hard cases. The participating systems need to identify the correct antecedent for a pronominal anaphora. The antecedent candidates share the same semantic class which makes the identification challenging. For instance, a question in the challenge might be:

\begin{addmargin}[25pt]{25pt}
The cat tried to climb in the box but got stuck because it was too big.\\ What was too big?\footnote{The example was taken from https://artistdetective.wordpress.com/}
\end{addmargin}

The pronoun \textit{it} could either refer to \textit{The cat} or \textit{the box}. The identification of the correct antecedent requires background information on sizes (``if A is in B, then B is bigger'') and additionally needs to identify that ``but got stuck'' indicates that the cat was too big. The implemented algorithm will most likely fail to identify the correct antecedent in those cases since it does not provide any kind of world knowledge.

\section{Outlook}

What can/has to be/may be done in future research? Impact on other branches of science? society?