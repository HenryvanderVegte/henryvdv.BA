%% +++++++++++++++++++++++++++++++++
%% Setzen von scrreprt
%% +++++++++++++++++++++++++++++++++
\documentclass[
11pt,
titlepage,
a4paper,
abstracton,
oneside,
openright,
chapterprefix,
noappendixprefix,
headsepline,
footsepline,
cleardoubleplain,
bibtotoc,
liststotoc,
pointlessnumbers,
article,
citestyle=authoryear 
]{scrreprt}



%% +++++++++++++++++++++++++++++++++
%% Einbinden von Paketen
%% +++++++++++++++++++++++++++++++++
\usepackage{pdfpages}
\usepackage{moresize}
\usepackage{tikz-dependency}
\usepackage{svg}
\usepackage{amsmath}
\usepackage{multirow}
\usepackage{units}
\usepackage{apacite}
\usepackage{caption}
\usepackage{istitle}
\usepackage{geometry}
\usepackage[utf8]{inputenc} 
\usepackage[T1]{fontenc}
\usepackage{ae,aecompl}
\usepackage{amsmath}
\usepackage{amsthm}
\usepackage{amscd}
\usepackage{amsfonts}
\usepackage{amssymb}
\usepackage{listings}
\usepackage{xcolor}
\usepackage{graphicx}
\usepackage{url}
\usepackage[automark]{scrpage2}
\usepackage{bbm}
\usepackage{array}
\usepackage{booktabs}
\usepackage{threeparttable}
\usepackage{pifont}
\usepackage{placeins}
\usepackage[font=small,labelfont=bf,labelsep=colon]{caption}
\usepackage{natbib}

%\bibliographystyle{abbrvnat}

% Settings for generating the PDF
% See https://www.tug.org/applications/hyperref/manual.html

\usepackage{hyperref}

\hypersetup{
	pdfinfo={ 
    		Title={Diplomarbeit},
		Creator={TeX},
		Producer={pdfTeX 0.15a},
		Author={},
		CreationDate={D:20091004000000},
		ModDate={D:20130331000000},
		Subject={Master thesis},
		Keywords={}
	},
	pdfpagelayout=TwoColumnRight,
	pdfdisplaydoctitle=true
}



%% Settings for generating the PDF
% See https://www.tug.org/applications/hyperref/manual.html


\hypersetup{
}


 % Links klickbar, vom PDF-Viewer hervorgehoben
%% Settings for generating the PDF
% See https://www.tug.org/applications/hyperref/manual.html


\hypersetup{
	colorlinks, % colored links
	linkcolor=blue,
	filecolor=darkgreen,
	urlcolor=red,
	citecolor=green,
	hypertexnames=false,
}
 % Links klickbar, farbig hervorgehoben
% Settings for generating the PDF
% See https://www.tug.org/applications/hyperref/manual.html


\hypersetup{
	hidelinks=true % for hiding links altogether
}


 % Links klickbar, aber nicht hervorgehoben



%% +++++++++++++++++++++++++++++++++
%% Header
%% +++++++++++++++++++++++++++++++++

% Seitengeometrie
\geometry{a4paper,outer=38mm,inner=26mm,top=40mm,bottom=50mm}

% Definition von arg max
\DeclareMathOperator*{\argmax}{arg\,max}

% Definition von Sternchen
\newcommand{\sig}{\ding{73}}
\newcommand{\ssig}{\ding{72}}

% Definition von Häkchen und Kreuzchen
\newcommand{\h}{\ding{51}}
\newcommand{\x}{\ding{55}}
\newcolumntype{M}{>{\centering\arraybackslash}m{\dimexpr.25\linewidth-2\tabcolsep}}

% Farben definieren
\definecolor{lightgrey}{rgb}{0.99,0.99,0.99}
\definecolor{colKeys}{rgb}{0,0,1}
\definecolor{colIdentifier}{rgb}{0,0,0}
\definecolor{colComments}{rgb}{1,0,0}
\definecolor{colString}{rgb}{0,0.5,0}

\definecolor{darkred}{rgb}{0.5,0,0}
\definecolor{darkgreen}{rgb}{0,0.5,0}
\definecolor{darkblue}{rgb}{0,0,0.5}
\definecolor{green}{rgb}{0,0.7,0}
\definecolor{blue}{rgb}{0,0,0.7}
\definecolor{red}{rgb}{0.7,0,0}
\definecolor{black}{rgb}{0,0,0}

% Quellcode
\lstloadlanguages{XML} 
\lstset{
    float=hbp,
    keywordstyle=\color{colKeys},
    stringstyle=\color{colString},
    commentstyle=\color{colComments},
    basicstyle=\texttt\small,
    identifierstyle=\color{colIdentifier},
    columns=flexible,
    tabsize=2,
    frame=single,
    extendedchars=true,
    showspaces=false,
    showstringspaces=false,
    numbers=none,
    numberstyle=\tiny,
    breaklines=true,
    backgroundcolor=\color{lightgrey},
    breakautoindent=true,
	captionpos=b,
	xleftmargin=\fboxsep,
	xrightmargin=\fboxsep,
	frameround=tttt
}

% Kopf- und Fußzeilen
\pagestyle{scrheadings}
\ihead[]{}
\chead[]{}
\ohead[]{\textsf{\headmark}}
%\ifoot[]{}
%\cfoot[]{}
%\ofoot[]{\textsf{\pagemark}}

% Absätze
\setlength{\parindent}{0pt}
\setlength{\parskip}{2ex}

\def\topfraction{1.0} 
\def\bottomfraction{1.0} 
\def\textfraction{0.0}

\renewcommand*{\partpagestyle}{empty}

% Überschrift des Abstracts
%\addto\captionsngerman{\renewcommand*\abstractname{Kurzzusammenfassung}}
% Überschrift des Verzeichnisses der Listings
%\addto\captionsngerman{\renewcommand*\lstlistlistingname{Auflistungsverzeichnis}}
% Name von Listings
%\addto\captionsngerman{\renewcommand*\lstlistingname{Auflistung}}


%% +++++++++++++++++++++++++++++++++
%% Start des Dokuments
%% +++++++++++++++++++++++++++++++++
\begin{document}

% Titelseite
\title{Improving Anaphora Resolution Through Corpus Mined Gender Information}
\author{Jan Henry van der Vegte}
\matriculationid{3008277} 
\studyProgram{Applied Cognitive and Media Science} 
\tutor{Professor~Dr.-Ing.~Torsten Zesch}
{Prof. Dr. rer. soc. Heinz Ulrich Hoppe}
\thesistype{Bachelorthesis}

\logo{figures/logo_uni.png}
\timeperiod{July 2016}{October 2016}
\pagenumbering{gobble}
\maketitle

% Erklärung
\pagestyle{empty}
\begin{center}
\Large{\textsf{\textbf{Erklärung}}}
\end{center}
\vspace{0.8cm}
Hiermit erkläre ich, dass ich die vorliegende Arbeit ohne fremde Hilfe selbstständig
verfasst und nur die angegebenen Quellen und Hilfsmittel benutzt habe. Ich versichere
weiterhin, dass ich diese Arbeit noch keinem anderen Prüfungsgremium vorgelegt habe.
\\[1cm]
...................................................................\\[0.2cm]
Jan Henry van der Vegte, Datum
\pagestyle{empty}

% Inhaltsverzeichnis
\pagestyle{scrheadings}
%\setcounter{tocdepth}{2}
\setcounter{page}{2}
\tableofcontents

% Text
\cleardoublepage
\pagenumbering{arabic}
\pagestyle{scrheadings}
\sloppy
\chapter{Introduction}
\label{sec:Introduction}

\section{Background}

In the last decades, the amount of textual information in media has increased severely, making automatic text comprehension indispensable. Since textual data found online is mostly unstructured, meaning that there is no formal structure in pre-defined manner, various information need to be added in order to make automatic understanding possible. For several natural language processing (NLP) tasks, referential relationships between words  in a document need to be set. 

The procedure of determining whether two expressions  refer to each other, meaning that they are instances of the same entity, is called anaphora resolution. The word to be resolved is termed anaphora, while its predecessor is the antecedent. It differs from coreference resolution by only resolving words, which can only be interpreted through its antecedent (Recasens et al., 2007) (1), while all corefering expressions are considered in coreference resolution (2).

(1) [Aberfoyle] describes [itself] as [The Gateway to [the Trossachs]]. \\
(resolve "itself" to "Aberfoyle")

(2) As late as 1790, all the residents in the parish of [Aberfoyle] spoke [Scottish Gaelic]. From 1882 [the village] was served by [Aberfoyle railway station].\\
(resolve "the village"  to "Aberfoyle")

Resolving noun phrases is a growing task in Natural Language Processing (NLP) and increased its relevance in the last decades, that it has even developed into a standalone subtask in the DARPA Message Understanding Conference in 1995 (MUC-6 1995). The International Workshop on Semantic Evaluation (SemEval) ran a coreference resolution task on multiple languages (Recasens et al., 2010), emphasizing the importance of coreference resolution systems. 
There are several important applications of coreference and anaphora resolution, such as Information Extraction (IE) (McCarthy and Lehnert, 1995), Question Answering (QA) (Morton, 2000), and Summarization (Steinberger et al., 2007).\\ 
Information Extraction has set itself the objective of summarizing relevant information from documents. Anaphora resolution is needed, because the sought entity is often referenced through different words (for instance personal pronouns). McCarthy and Lehnert described it as a classification problem: "given two references, do they refer to the same object or different objects."\\
The Question Answering task described by Morton has the goal to find a 250 byte string excerpt out of a number of documents as the answer to a query. Annotated coreference chains were used to link all instances of the same entity in a document. Occurrences in an other sentence are given a lower weight for prediction. The use of annotated coreference chains improved the prediction slightly.\\
Steinberger et al. figured out that the additional use of anaphoric information improved their performance score over solely Latent Semantic Analysis (LSA) summarization.

A lot of different information need to be used since selecting a possible antecedent is a decision under high ambiguity. The decisive factor for determination might be for instance gender-, or grammatical number agreement . Sometimes there is no decisive factor at all. Examples for gender agreement are shown in (3) and (4), for number agreement in (5) and (6).

(3) John and Jill had a date, but he didn't come. (resolve "he" to "John").

(4) John and Jill had a date, but she didn't come. (resolve "she" to "Jill").

(5) John loves his children. They are very nice. (resolve "they" to "his children").

(6) John loves his children. He is very nice. (resolve "he" to "John").


There are lots of different approaches for coreference and anaphora resolution, leading from rule-based techniques to machine learning. Recasens et al. indicated, that 

\section{Motivation}


Due to the difficulty of coreference and anaphora resolution, there are countless approaches, from rule-bases techniques to machine learning. 

In this work i will present a machine learning approach 
Pronoun resolution was contemplated in the SemEval coreference resolution task. 




establish your territory (say what the topic is about) and/or niche (show why there needs to be further research on your topic)
shortly introduce your research/what you will do in your thesis (make hypotheses; state the research questions)



\chapter{Related Work}
\label{sec:Related Work}

Anaphora resolution systems emerged into two different strategies. First of all, there are rule-based techniques which focus more on theoretical considerations. The second strategy uses machine learning and is based on annotated data. The following chapter will briefly present both and discuss their advantages and disadvantages, followed by exemplary realisations. Since anaphora resolution is a subtask of coreference resolution, coreference resolution systems will be considered as well.

\section{Rule-Based Techniques}
Rule-based techniques rely on manual understanding and implementation of syntactic and semantic principles in natural language \citep{kennedy1996anaphora,mitkov1994integrated,ingria1989computational}. Clues that could be helpful for antecedent identification are manually implemented as rules. To identify relevant clues, prior knowledge about linguistic principles (such as binding principles) is necessary. Since rules might be domain-specific, the implementation would most likely be worse on other domains. Refinements for different domains would make the development even more complex and time-consuming. Nevertheless, rule-based techniques are much more transparent in contrast to machine learning. In the last section, a comparing evaluation of both techniques will be presented.

\subsection{The Naive Hobbs algorithm}
The Naive Hobbs algorithm described by \citep{hobbs1978resolving} relies on parsed syntax trees containing the grammatical structure. Put simply, the tree containing the anaphora is searched left-to-right with breadth-first search and the algorithm stops when a matching noun phrase is found. Noun phrases mismatching in gender or number are neglected. The algorithm also limits the list of possible antecedents, as for instance the antecedent can not occur in the same non-dividable noun phrase. As long as no matching antecedent is found, the preceding sentence will be searched successively. \\
Hobbs reported an accuracy score of 88.3 \% on the pronouns “he",“she",“it", and “them" with only using the algorithmic approach. The usage of additional constraints improved the accuracy to 91.7 \%.

\subsection{CogNIAC}
Another rule-based approach was presented by \citep{baldwin1997cogniac} with CogNIAC, a high precision pronoun resolution system. It only resolves pronouns when high confidence rules (shown in Table \ref{table:cogniacRules}) are satisfied in order to avoid decisions under ambiguity and to ensure that only very likely antecedents are attached (high precision). This might lead to a neglect of less likely but still correct antecedents and lower the recall score. 
\\
For each pronoun the rules are applied one by one. If the given rule has found a matching candidate it will be accepted. Otherwise the next rule will be applied. If none matches the candidates it will be left unresolved as this implicates a higher ambiguity. In order to apply Baldwins high confidence rules, information on sentences, part-of-speech, and noun phrases is required and therefore annotated. Semantic category information such as gender and number is determined through various databases. Confirming their prediction, \citep{baldwin1997cogniac} reported a high precision score (97 \%), but lower recall (60 \%) on their training data consisting of 198 pronouns.\\
As can be seen the order of rules lead from higher to lower precision: if only one possible antecedent can be found (rule 1) it is most likely the correct antecedent while rule 6 indicates more ambiguity as it relies on more content-related information. Human understanding of syntax and semantics is needed to determine a specific order of rules. Therefore, adding new rules might not improve the performance even though those rules are reasonable in itself. Most rule-based systems struggle with that problem.\\
In a second evaluation, CogNIAC was compared to the Hobbs Algorithm \citep{baldwin1997cogniac,hobbs1978resolving} on singular third-person pronoun resolution. In order to maximize the ambiguity, the training data texts were narrations about same gender characters. To make accuracy scores comparable, \cite{baldwin1997cogniac} added lower precision rules, such as the most recent antecedent should be picked if no other rule found a matching noun phrase. The Accuracy scores reported were nearly equal (78.8\% on the Hobbs Algorithm, 77.9\% on CogNIAC), underlining the reason of existence of various approaches.

\begin{table}[h]
    \begin{tabular}{| l |p{8cm} |}
    \hline
    Rule & Description \\ \hline
\hline
    1) Unique in Discourse & If there is a single possible antecedent PAi in the read-in portion of the entire discourse, then pick PAi as the antecedent. \\ \hline
    2) Reflexive & Pick nearest possible antecedent in read-in portion of current sentence if the anaphora is a reflexive pronoun \\ \hline
    3) Unique in Current + Prior & If there is a single possible antecedent i in the prior sentence and the read-in portion of the current sentence, then pick i as the antecedent: \\ \hline
    4) Possessive Pro & If the anaphora is a possessive pronoun and there is a single exact string match i of the possessive in the prior sentence, then pick i as the antecedent:  \\ \hline
    5) Unique Current Sentence & If there is a single possible antecedent in the read-in portion of the current sentence, then pick i as the antecedent  \\ \hline
    6) Unique Subject/ Subject Pronoun & If the subject of the prior sentence contains a single possible antecedent i, and the anaphora is the subject of the current sentence, then pick i as the antecedent \\ \hline
    \end{tabular}
  \caption{CogNIAC core rules}
     \label{table:cogniacRules}
\end{table}

\subsection{Anaphora Resolution with Limited Knowledge}
\label{anaphoraLimitedKnowledgeSection}

A domain independent approach by \cite{mitkov1998robust} tried to eliminate the disadvantages of previous rule-based systems. Mitkov renounced complex syntax and semantic analysis in order to keep the algorithm as less domain specific as possible. Only a part-of-speech tagger and a simple noun phrase identifitcation module were applied. The algorithm was informally described by Mitkov in three steps:
\begin{enumerate} 
\item Examine the current sentence and the two preceding sentences (if available). Look for noun phrases only to the left of the anaphora
\item Select from the noun phrases identified only those which agree in gender and number with the pronominal anaphora and group them as a set of potential candidates
\item Apply the antecedent indicators to each potential candidate and assign scores; the candidate with the highest aggregate score is proposed as antecedent
\end{enumerate}

Overall, a set of 10 antecedent indicators were used which indicate either a high or a low likelihood for the noun phrase to be the antecedent. Negative indicators such as definiteness (whether the noun phrase contains a definite article, whereby indefinite phrases decrease the likelihood) and positive indicators like term preference (if the noun phrase is a term in the field, the likelihood is increased). The score values are integers from -1 to 2. \\
Mitkov reported a success rate of 89.7 \% on random sample texts of technical manuals. A modified approach could also be applied for polish \citep{mitkov2000robust} and arabic \citep{mitkov1998multilingual} with similar success rates.
A comparing evaluation to Baldwins CogNIAC \citep{baldwin1997cogniac} indicated a superiority of Mitkovs approach \citep{mitkov1998robust} as CogNIAC had a lower success rate of approximately 15 \% on the previously described data set. The stated reason for the comparison was that the approaches showed several similarities as both require few preprocessing and gain their information mostly from part-of-speech tags and noun phrases.\\
The superiority of \citep{mitkov1998robust} could be explained by its handling of uncertainty as the antecedent indicators are not implemented as hard constraints. Basically, Mitkovs anaphora resolution system can be described as a combination between rule-based and statistical techniques in order to use the best of both worlds.

In 2002, a revised version of the original approach by Mitkov was presented \citep{mitkov2002new}. The improved version of the original algorithm called MARS had some smaller and greater changes:\\
First of all, three new antecedent indicators and a module for identification of pleonastic pronouns\footnote{A pleonastic pronoun is non-referential. For example the \textit{it} in “it is raining" } and non-nominal pronominal anaphoras were added. Additionaly, the implementation of some previous features was changed as other preprocessing tools were used.

\subsection{GuiTAR}
With GuiTAR, a modular anaphora resolution tool was developed \citep{poesio2004general}. It was designed to be domain-unspecific and usable off-the-shelf which means that preprocessing steps such as part-of-speech tagging and named entity recognition will be added on itself. Either raw text data or XML files can be used as the input. In case of raw text data, XML files with annotated part-of-speech tags, noun phrase boundaries, pronoun categories etc. will be created. The anaphora resolution system relies on Mitkovs MARS-algorithm \citep{mitkov2002new}, which was introduced in section \ref{anaphoraLimitedKnowledgeSection}. 

\citep{poesio2004mate} reported an F-measure of 64.2 \% for personal pronouns on raw text data of the GNOME corpus \citep{poesio2004general}. In comparison, the baseline approach (choosing the most recent antecedent) achieved an F-measure of 50.5 \% on the same data.

%vllt noch ergänzen für andere Sprachen
\section{Machine Learning-Based Techniques}

Most machine learning-based techniques learn principles from annotated text corpora \citep{soon2001machine, bergsma2005automatic} which include the correct label for each instance. In this context, a label will contain the information whether a noun phrase is the antecedent. A decisive factor of machine learning is that irrelevant information (presented through features) has a lower impact on success factors (the accuracy for instance) compared to rule-based techniques, as the algorithm automatically learns to rate those as irrelevant and vice versa. Therefore, machine learning approaches tend to have little information on linguistic principles as the algorithm should learn those autonomously. This causes the algorithm to be less domain specific, but increases the risk to miss relevant clues. However, top-performing machine learning approaches achieve accuracy scores comparable to best non-learning techniques \citep{soon2001machine}. \\
Additionally, machine learning algorithms are usually more time-consuming due to the learning process.

\subsection{Anaphoras in Coreference Resolution}
\label{soon2001traininginstances} %falls später darauf verwiesen werden soll wie die Instanzen erstellt werden

As already stated, coreference resolution aims for linking all noun phrases referring to the same entity in the real world in a document. The most common kind of storing coreferential information is through coreference chains, in which the current element always points towards the following same entity-element. Another way of storing coreferences is to define a unique ID for each real-life entity. All occurences in the text will be assigned to their belonging IDs.

An often quoted coreference resolution system using machine learning was proposed by \cite{soon2001machine}. In this case decision trees was chosen as a classifier. A natural language processing pipeline was used for the identification of markables. The pipeline identified amongst others part-of-speech tags, noun phrases, named entities, and semantic classes. A high value was placed on designing generic features to make them domain-independent. In total, a set of 12 different features was used. It covers inter alia a distance feature (standing for the distance in sentences between two elements), a gender agreement feature (whether the gender matches), and a number agreement feature (whether the number matches). Deriving gender information of a noun requires information of their semantic classes. \cite{soon2001machine} worked with the simplified assumption that the semantic class of a noun phrase is the semantic class of the most frequent sense of the considered noun in WordNet. Gender agreement was assumed if both phrases got the same semantic class (for example “male") or if one is the parent of the other (such as phrase one is considered as “person" and phrase two as “male"). 
In order to make machine learning possible, training instances need to be generated.\\
To generate positive training instances \cite{soon2001machine}. used every noun phrase in a coreference chain and its predecessor in the same chain. Each intervening noun phrase forms a negative instance with the considered noun phrase. 

The researchers reported an F-measure of 62.6 \% on the MUC-6 data and comparable results on the MUC-7 data. A comparison with official MUC-scores indicated, that their system performed at the upper bound of the considered systems. Those values and the used feature set are often referred as baseline for further systems \citep{versley2008bart}.

\citep{ng2002improving} extended their work and improved it through additional features, a different training set creation, and a clustering algorithm to find the noun phrase with the highest likelihood of coreference. The majority of the new features is based on syntactical principles. For instance, binding constraints must be fulfilled and one phrase is not allowed to span another. Positive training instances are not created through their preceding antecedent, but through their most confident one. In addition, they started to search for a related antecedent from right-to-left for a highly likely antecedent (in contrast to starting the right-to-left search for the first previous noun phrase). Ng and Cardie reported a significant increase in precision and F-measure compared to the initial approach by \citep{soon2001machine}.

\subsection{BART}
\nocite{versley2008bart}
In 2008, Versley et al. introduced a coreference resolution system for raw text data which extended the previously described approach by \cite{soon2001machine}. The ambition for BART was to keep it as modular as possible so that it could be applied to many different subtasks of coreference resolution. BART consists of a preprocessing pipeline for parsing, part-of-speech tagging, and further basic information and a mention factory for mainly gender and number identification. Additionaly, a feature extraction module and therefore a matching decoder and encoder is included. The decoder generates the training data while the encoder prepares the testing data. Similar to \citep{soon2001machine} the feature labels are binarized which means that an anaphora either contains the correct or wrong antecedent. Accordingly, the feature labels are either true or false. \\
A subsequent approach on multiple languages with BART \citep{broscheit2010bart} used a feature set of seven features for all classification types, including a gender agreement, number agreement, string match, and distance feature. The procedure of gaining gender and number information was adopted by \cite{soon2001machine}. 

An F-measure of approximately 55.6 \% on Bnews articles of the ACE-2 corpora was reported with the usage of the basic feature set \citep{versley2008bart}. 
With additional language-dependent features, BART was successfully transferred to german \citep{broscheit2010extending}, polish \citep{kopec2012creating}, and italian \citep{poesio2010creating}.

\citep{reiteretal:2011b} indicated that a great weakness of BART is the implementation of gender information as in their evaluation even noun phrases with explicit gender information were linked incorrectly.

\subsection{Cluster-Based Coreference Resolution}
The previously described machine learning approaches generate negative and positive training examples as pronoun-antecedent pairs. \cite{rahman2009supervised} pointed out several disadvantages of pairwise comparisons: First of all, each possible antecedents is considered on its own which makes a comparison between candidates impossible. For instance, if the first preceding candidate was accepted as the antecedent because it passed a defined threshold, no further candidate will be analysed even if it passed the threshold with a much higher value. Secondly, several contextual information might be missing as only the pronoun-antecedent pair is examined. Those contextual clues could inter alia give information on gender or number. 
In order to solve those disadvantages \cite{rahman2009supervised} presented a cluster-ranking coreference module. A cluster ranker is trained to determine to which previous coreference cluster a coreference should be resolved. In contrast to rule-based approaches \citep{mitkov1998robust} no manual constraints restrict possible candidates. Instead, restrictions are learned through features automatically. \cite{rahman2009supervised} implemented three different classifiers baselines which represent previous learning-based approaches to coreference resolution. For instance, a mention-pair coreference model was implemented. This classifier learns from coreferent-anaphora-pairs which either conatin the correct or wrong antecedent (therefore this approach is similar to \cite{soon2001machine}). 
The implemented cluster ranker!!!!!!!!!!!

When evaluated on the ACE 2005
coreference data sets, cluster rankers outperform
three competing models — mention-pair, entitymention,
and mention-ranking models — by a
large margin.

\subsection{Pronoun Resolution in Spoken Dialogue}
As already mentioned, machine learning approaches are less domain-specific than rule-based systems. For that reason \citep{strube2003machine} presented an corpus-based approach for pronoun resolution in spoken language. Still, several extensions and adaptions had to be done as spoken dialogue differs from written texts gravely. Firstly, the number of pleonastic pronouns in spoken dialogue is substantially increased. Secondly, a not ignorable amount of anaphoras in spoken dialogue dont have a clearly defined antecedent so that even humans cant determine them. \citep{eckert2000dialogue} called them vague anaphoras and figured out that 13.2 \% of all anaphoras in their examined corpus fall in that category.\\
A corpus of twenty switchboard dialogues was used. In order to generate training data, a list of all potential anaphoras was created. Potential anaphoras are all non-definite noun phrases except for first and second person pronouns. Each element in the remaining list forms a pair with every preceding noun phrase that does not disagree in gender, number, or person. If the instances corefer they were labelled P, else N. For all anaphoras without explicit noun phrase antecedents other phrases (for instance verb phrases) in the current last two sentences were used to form pairs. \\
The feature set with a total of 25 features included noun-phrase features, coreference-level features and spoken dialogue features. Noun-phrase features rely on further preprocessing such as gender, number, or the grammatical function of the anaphora or the antecedent. Coreference-level features could be described as low-level preprocessing features. Those features mainly describe the distance between the antecedent and the anaphora, for instance in words or sentences. The features especially for spoken dialogue contain for instance information on how many noun phrases are located between anaphora and antecedent.
A decision tree classifier with 20-fold crossvalidation was applied. \citep{strube2003machine} reported an F-measure of 47.42 \% for the full classifier, including all pronouns and all features. 


\subsection{Corpus- and Web-Mined Gender Information}
\label{section:bergsma2005automatic}
\cite{bergsma2005automatic} presented a machine learning approach to anaphora resolution which treats gender information not as a hard constraint, but as a probability distribution of possible outcomes. A majority of previous approaches assigned either a specific gender and number (e.g. masculine, feminine, neutral, or plural) or, in case of uncertainty, no gender at all \citep{soon2001machine, broscheit2010bart}. Another motivation was that \cite{kennedy1996anaphora} reported to attribute 35 \% of their resolution errors to gender mismatch. Only third-person pronouns were considered.

The gender information was derived of two sources: a text corpus and the web. 
For the former, all occurences of nouns and pronouns in lexico-syntactic patterns are counted. Five different patterns for reflexives, possessives, nominatives, predicates, and designators were used (Table \ref{table:bergsma2004GenderTable}). A reflexive masculine occurence would be for instance “John likes himself". In this case, a counter for “John" with masculine gender and reflexive pronoun will be increased. This procedure was repeated for all other patterns and remaining genders and numbers (masculine, feminine, neutral, and plural). \cite{bergsma2005automatic} applied lots of textual data in order to offset parser errors and other noise sources. The whole data set included the AQUAINT corpus \citep{graff2002aquaint} as well as the Reuters corpus \citep{rose2002reuters}. In total, a data set of approximately 6 gigabytes of text was used.

\begin{table}[h]
    \begin{tabular}{| l | p{5cm} | p{5cm} |}
    \hline
    Gender Corpus Indicators & Contained Elements & Pattern \\ \hline
\hline
    1) Reflexive & himself, herself, itself, and themselves &  \textit{noun} + \textit{verb} + \textit{reflexive}\\ \hline
    2) Possessive & his, her, its, and their & \textit{noun} + \textit{verb} + \textit{possessive} + \textit{noun} \\ \hline
    3) Nominative & he, she, it, and they & \textit{noun} + \textit{verb} + \textit{nominative} +  \textit{verb} \\ \hline
    4) Predicate & he, she, it, and they & \textit{pronoun} + is/are [a] + \textit{noun}  \\ \hline
    5) Designator & Mr. and Mrs. & \textit{designator} + \textit{noun}\\ \hline
    \end{tabular}
  \caption{Gender Corpus Patterns}
     \label{table:bergsma2004GenderTable}
\end{table}


Since a text corpus, no matter how big it is, cant contain all possible words and word combinations, the web was used as a second information source. The Google API was used to count the web pages that appear if a noun, the Google wildcard operator (“*"), and the gender indicator were searched. For instance, if the gender of “John" should be determined, a Google request will be sent with all gender indicating elements of Table \ref{table:bergsma2004GenderTable} ( John * himself, John * herself, John * itself, etc.).
In the following step, the probabilities for each gender will be determined through the five corpus sources and the five web sources. The naive approach would be that the probability of the indicator to be masculine is the percentage of all cases in that the word occurs with its masculine indicator. For instance, in Table \ref{table:bergsma2004GenderFreqTable} the cumulated frequency of "Alex" occuring with “himself" is 60. In total, “Alex" was found 100 times with a reflexive pronoun. As a consequence, the probability for “Alex" to be masculine would be estimated at 60 \% from reflexive indicators.
This approach leads to three major problems. First of all, zero-probabilities would indicate that there is no possibility for noun to belong to that gender. This might be true - some words might never be part of a certain gender. On the other hand, however, it might just be a rare event and an occurrence would be found with a larger or different text corpus. Secondly, adding a further count could change the likelihood enormous for small frequencies. This leads to the third problem: a measure is needed to determine the certainty of a likelihood. A 70 \% probability of a word to be masculine is more meaningful if 1000 cases are considered rather than 10. 
 
In order to solve those problems, \cite{bergsma2005automatic} treated the counts as a Beta distribution in a Bayesian approach. More precisely, two parameters named $\alpha$ and $\beta$ are considered. For each gender, $\alpha$ determines the count of the considered event plus one (in order to avoid zero-probabilities) while $\beta$ represents the count of all not considered events plus one. The $\alpha$ and $\beta$ values of the previous “Alex" example with reflexive indicators for masculine gender would be $\alpha$ = 61 and $\beta$ = 41. The mean value of it is computed as:
\begin{center}
	 $\mu$ =  $\dfrac{\alpha }{\alpha + \beta}$  
\end{center}

A complete distribution is presented in Table \ref{table:bergsma2004GenderFreqTable}. Note that, unlike the naive approach, these values can only be partially compared to one another, as each of them represents a single distribution. Furthermore, the percentages dont even sum up to 100 \%.

\begin{table}[h]
    \renewcommand{\arraystretch}{2.0}
    \begin{tabular}{| l | l | l | l |}
    \hline
    Gender/Number & Occurences & Naive Approach & Bayesian Approach \\ \hline
\hline
    1) Masculine & 60 & $\dfrac{60}{100}$ = 60 \% &  $\dfrac{61}{102}$ $\approx$ 59.8 \% \\ \hline
    2) Feminine & 30 &  $\dfrac{30}{100}$ = 30 \% &  $\dfrac{31}{102}$ $\approx$ 30.4 \% \\ \hline
    3) Neutral & 10 & $\dfrac{10}{100}$ = 10 \% & $\dfrac{11}{102}$ $\approx$ 10.8 \%  \\ \hline
    4) Plural & 0 &    $\dfrac{0}{100}$ =  0 \% &  $\dfrac{1}{102}$ $\approx$ 0.1 \% \\ \hline
    \end{tabular}
      \caption{Gender Frequencies Example}
     \label{table:bergsma2004GenderFreqTable}
\end{table}



 \cite{bergsma2005automatic} expressed the certainty through the variance of Beta distributions:

\begin{center}
	$\sigma$\textsuperscript{2} =  $\dfrac{\alpha \beta }{(\alpha + \beta)\textsuperscript{2}(\alpha + \beta + 1)}$
\end{center}

In case of little or no counts at all, the variance will be approximately 1/12. The classifier should automatically learn that distributions with that variance wont be meaningful.

In order to prove the accuracy of their gender classification, \cite{bergsma2005automatic} built several SVM-Classifiers. Overall, a set of 20 features was used: Each of the five gender indicators (reflexive, possessive, etc.) has its mean and its variance as features (in this case, the standard deviation was used which is the square root of the variance). Each of the gender indicators was implemented corpus-based as well as web-based. All gender features led to an F-measure of 92 \%. Seperate classifiers for either web-based or corpus-based information yielded to an F-measure of 85.4 \% for the corpus-based and 90.4 \% for the web-based approach.

Various pronoun resolution classifiers were built in order to determine the influence of several aspects. In general, each classifier searches, beginning by the certain anaphora, the text backwards until a matching antecedent is found. The matching criteria vary depending on the complexity of the classifier. The search backwards of the more complex classifiers was limited so that only the current and the previous sentence was considered, because a corpus observation showed that more than 97 \% of all antecedents could be found in that range. If no accepted antecedent was found a threshold was reduced so that antecedents with lower likelihood might be accepted. This procedure was repeated until the first candidate exceed the threshold. 

The first baseline approach was to always select the most recent noun phrase as antecedent. An accuracy of 26.0 \% was reported. \\
A first improvement consisted of the use of only explicit gender indicators such as “Mr." and “Mrs." to determine the gender. The first previous antecedent that does not mismatch will be chosen. The accuracy was improved up to 30.8\%. 
In a third baseline approach the previously mentioned gender SVM-classifiers were used to detect a gender match or mismatch. Underlining the importance of gender and number agreement, the accuracy rose up to 59.4 \%.

The first machine learning approach included a feature set of 39 features, whereby most of the features were binarized. The features can be seperated into three categories. First of all, there are pronoun-related features that determine the gender and number of the pronoun. Secondly, antecedent-related features which provide for instance information on the grammatical relation of the noun phrase or whether it is a person or an organization. The third group of features describes the relation of pronoun and antecedent and contains features that rely on linguistic principles (such as if binding principles are satisfied) as well as features that only require basic preprocessing steps (sentence and word distance for instance). In order to apply those, the texts were tokenized, parsed, and noun phrases were linked. The training instance creation procedure was adopted by \cite{soon2001machine} and was previously described in Section \ref{soon2001traininginstances}. In total, 1251 positive and 2909 negative training instances were created. 

The classifier reached a performance score of 62.3 \% which is above all previous approaches. The additional use of corpus and web frequency features and three other gender affecting features led to a performance score of 73.3 \%. 


\section{A Comparison of Both Strategies}
\cite{aone1995evaluating} did a comparison of a previously build manually designed resolver (MDR) \citep{aone1993language} and their in 1995 introduced machine learning-based resolver (MLR).\\
This section will briefly explain both implementations in order draw an appropriate conclusion of the comparison. % // to make the comparison more interpretable y comparison //illustrate //allow room for interpretation

\subsection{A Manually Designed Resolver (MDR)}
The manually designed resolver was build to be language-independent, extensible, robust, and tunable for specific domains. The used information was derived through three different knowledge bases: the \textit{Discourse Knowledge Source}, the \textit{Discourse Phenomenon}, and the \textit{Discourse Domain}. \\
The former contains antecedent generators to determine all possible antecedents, a system to filter out unwanted antecedent candidates, and an orderer to rank the candidates from highest to lowest likelihood. All of these components rely on specific rules and functions. For instance, the filter removes candidates of mismatching gender. Even though some the rules are only applied on specific languages, \citep{aone1993language} reported that most of them are language-independent.\\
The \textit{Discourse Phenomenon} contains all possible part-of-speech categories in which the anaphora could occur in a hierarchical order. For instance, “third-person" pronoun is a subclass of “pronoun". Each class includes its definition, two resolution strategies (a second one is needed if the main strategy fails), and specific language information if a category only exists in a certain language.\\
The third knowledge base is responsible for domain-specific information. 

A module called \textit{Discourse Administrator} was used to determine the application domain and in a further step to select and filter the knowledge bases in order to generate the best possible resolution system. Therefore, the information stored in each knowledge base is heavily dependent on the considered language and domain. The general resolution process is as follows: The discourse phenomena are used to determine all anaphoras. In a second step, the discourse knowledge sources are applied in order to generate and filter all possible candidates. If only one remains, it will be chosen as antecedent. Otherwise, one or more orderers are applied and the best candidate will be chosen by order. If no candidate was found at all, the second strategy specified in the discourse phenomenon will be applied.


\subsection{A Machine Learning-Based Resolver (MLR)}
The machine learning-based resolver presented by \cite{aone1995evaluating} used pairwise training examples containing information on the anaphora and its possible antecedent. A whole set of 66 features was used. \cite{aone1995evaluating} divided most of them into one of four subcategories, namely lexical, syntactic, semantic, and positional. The feature selection inspired by the manually designed resolver \citep{aone1993language}, but were generalized and changed in order to be domain- and language independent.

In total, six different classifiers depending on three parameters were trained. The first parameter was called anaphoric chain. If its value is true, a correct antecedent is detected if the candidate is part of the same anaphoric chains which means that both refer to the same real-world entity. Otherwise, just the preceding same-world entity will be accepted as correct antecedent. This parameter also affects the training instance generation. In case of anaphoric chains, all co-refering phrases will form positive training instances with its anaphora. In the other case, just the preceding co-refering phrase will be used for positive instances. In both cases, the remaining phrases will form negative training instances with the anaphora. The second parameter determines whether the decision tree will use further information of the anaphoric type (for instance whether the real-world entity of the anaphora is a proper name). A third parameter determines the pruning-factor of its decision tree. A high pruning-factor indicates a higher generalization while decision trees with a lower factor tend to overfit. 

\subsection{Evaluation}
The comparison was evaluated on japanese newspaper articles. In total, 1271 anaphoras were used. As it can be seen in Table \ref{table:aone1995evals}, all machine learning approaches using anaphoric chains outperformed the manual approach independent of their pruning-factor, while the approach without the usage of anaphoric chains performed slightly worse. The different pruning factors seemed to have a rather low impact on the performance.

As the manually designed resolver also detects only the preceding same-world entity, it would be most reasonable to compare it to the MLR-6. Even though the manual approach performed better, no language specific information or relevance of features need to be determined as the algorithm learned it autonomously \citep{aone1995evaluating}. \cite{aone1995evaluating} interpreted the results as optimistic for machine learning techniques.

%Zero-pronouns!

\begin{table}[h]
\begin{tabular}{|c|c|c|c|c|}
	\cline{1-5}
	Algorithm & Anaphoric Chains & Anaphoric Type& Confidence & F-measure \\ \cline{1-5}
	\cline{1-5}
	MLR-1 & yes & no & 100 \% &  76.27 \\ \cline{1-5}
	MLR-2 & yes & no & 75 \% & 77.30 \\ \cline{1-5}
	MLR-3 & yes & no & 50 \% & 76.43 \\ \cline{1-5}
	MLR-4 & yes & no & 25 \% & 77.28 \\ \cline{1-5}
	MLR-5 & yes & yes & 75 \% & 74.54 \\ \cline{1-5}
	MLR-6 & no & no & 75 \% & 67.03 \\ \cline{1-5}
	\multicolumn{5}{r}{}\\ \cline{4-5}
	\multicolumn{2}{r}{} & &  MDR & 69.57 \\
\cline{4-5}
	\end{tabular}
  \caption{Aone \& Bennett Evaluation}
     \label{table:aone1995evals}
\end{table}


\chapter{Data}
\label{sec:Data}

Most machine learning approaches require annotated corpora in order to make the learning process possible. In this case, the training corpora must contain information on the correct antecedent for each anaphora. Since this work is designed to learn gender information through frequencies, a second information source is needed. This chapter will briefly describe both information sources.

\section{WikiCoref }
\label{wikicorefSec}
\cite{wikicoref2016} presented with WikiCoref a coreference-annotated corpus of english Wikipedia articles. Wikipedia differs from most web corpora as it is highly structured. The Wikipedia guidelines\footnote{https://en.wikipedia.org/wiki/Wikipedia:Manual\_of\_Style} contain various restrictions on grammar and vocabulary and also define the structure of articles in terms of sections and paragraphs. In contrast, most other web-mined sources are heavily unstructured and could contain colloquial language as well as ungrammatical texts. \\
An excerpt of 30 articles was used to build the corpus. \citep{wikicoref2016} figured out that more than 35 \% of all Wikipedia articles contain less than 100 words and only 11 \% more than 1000 words. Articles with few word counts (less than 200 words) were not considered as they do not contain enough information for meaningful coreference resolution. Hence articles cannot be chosen completely random, a uniform distribution of categorized article sizes leading from less than 1000 to more than 5000 words was strived. Additionally, articles with too many out links were not considered. In order to keep the corpus domain-independent, articles of different topics were selected.\\

To detect entities, a combination of a coreference resolution system, an entity detection module, and anchored links in the article was used. The coreference chains detected by the module were manually corrected and missing ones were added. All coreferring entities were linked through a joint identification number representing the real-world entity.

\begin{addmargin}[0.5cm]{0.5cm}
[Aberfoyle]\textsubscript{1} is [a village in the region of Stirling, Scotland, northwest of [Glasgow]\textsubscript{2}]\textsubscript{1}.

[The town]\textsubscript{1} is situated on [the River Forth]\textsubscript{3} at the base of [Craigmore]\textsubscript{4} (420 metres high).
\end{addmargin}

In total, the corpus contains 59652 tokens\footnote{``A token is an instance of a sequence of characters in some particular document that are grouped together as a useful semantic unit for processing." (http://nlp.stanford.edu/IR-book/html/htmledition/tokenization-1.html)} in 2229 sentences with an average of 2000 tokens per document. For the inter-annotator agreement an MUC F1 score \citep{vilain1995model} of 83,3 \% was reported.

\section{Gender Corpus}
\label{section:gendercorpus}
An automatic approach to learning gender information through corpus- and web-based frequencies was introduced by \cite{bergsma2005automatic} and explained in Section \ref{section:bergsma2005automatic} of this work. 
\cite{Bergsma:06} pointed out two disadvantages of the previous approach. First of all, sending Google requests for each possible antecedent on large corpora is time-consuming and therefore not cost-efficient. Secondly, the corpus and web based implementations are not symmetric as some occurrences can only be found with the web-based approach. For instance, the corpus-based approach merely accepts a verb between a noun and a reflexive pronoun while the Google wildcard operator (``*") is not limited to any grammatical category.
Therefore, a new corpus mined frequency distribution of gender and number information was mined using a corpus of approximately 85 GB. Overall, an accuracy of 90,3 \% on gender determination was reported. \cite{Bergsma:06} made the mined gender and number frequencies openly accessible for the NLP community.\footnote{Available for download at http://www.clsp.jhu.edu/\textasciitilde sbergsma/Gender/Data/}
\chapter{Methodology}
\label{sec:Methodology}

\section{Featureset}

\subsection{Pronoun Features}

\subsection{Antecedent Features}

\subsection{Pronoun-Antecedent Features}

\subsection{Gender Features}

\section{Generating Training Instances}

\section{Baseline Approach}

\section{SVM Classifier}
\chapter{Evaluation}
\label{sec:Evaluation}

This chapter will present different performance measures and comparing evaluations in order to draw an appropriate conclusion. Additionally, a closer look at the errors made by the implemented system will give a hint on possible improvements.

\section{Results}
A precision measure indicates in how many percent of all cases the classifier identified the correct antecedent relative to the sum of all correct identified and falsely neglected antecedents. A recall measure indicates the amount of all correct identified cases relative to all identified cases. The F-measure is a balanced mean of both values.

The baseline achieved an accuracy of 26 \%, just like what \cite{bergsma2005automatic} reported as a baseline on the American National Corpus \citep{ide2001american}. Calculating the precision is not useful in this case as the baseline does not neglect any cases. Therefore, the precision would always be 100 \%. The recall measure is identical to the accuracy in that case.

\begin{table}[h]
\centering
  \caption{Pronoun resolution performance scores}
\scalebox{0.9}{
\begin{tabular}{|l|c|c|c|c|}
	\hline
	Method & Accuracy & Precision & Recall & F-measure \\ \hline
	\hline
	SVM Classifier (without corpus gender) & 61.9 \% & 95.8 \% & 63.6 \% & 76.5 \% \\ \hline
	SVM Classifier (with corpus gender frequencies) & 63.4 \%  & 92.6 \% & 66.8 \% & 77.6 \% \\ \hline
	SVM Classifier (with corpus gender constraints) & 64 \%  & 92.6 \% & 67.4 \% & 78 \% \\ \hline
	\end{tabular}
}
     \label{table:pronounResScores}
\end{table}


Table \ref{table:pronounResScores} shows the results of the SVM classifiers. All results were calculated with a 10-fold-crossvalidation on all 30 documents. The whole resolution system with all features received an F-measure of 76.5 \% and an accuracy of 61.9 \% which is more than twice as much as the baseline accuracy, indicating the clear benefit of the implemented system. A comparison with other anaphora resolution systems is only partially reasonable since different data sets and implementations were used. Still, the high precision rule-based approach presentend by \cite{baldwin1997cogniac} in Section \ref{baldwinCogNIAC} received similar precision and recall measures on the MUC-6 data set \citep{grishman1996message}. The approach of \cite{aone1995evaluating} described in Section \ref{aoneBennetEval} achieved an F-measure of approximately 77 \% for their best machine-learned classifiers on Japanese newspaper articles.

The corpus mined gender frequencies had only a slight impact on the classifiers performance, but caused an increased accuracy and F-measure. In order to associate the increase to the gender frequencies, a third classifier was built. It used the whole feature set except for gender means and standard deviations. Four new variables were added instead. Each variable represents a gender. The corpus frequencies were used to determine the most frequent gender of an antecedent candidate. The variable representing the most frequent outcome was set to one while the others were set to zero. For instance, \textit{Peter} was found 4479 times masculine, 76 times feminine, 81 times neutral, and 120 times plural. Since 4479 is the highest, the new variable representing the masculine gender will be set to one while all other new variables will be set to zero. This hard constraint implementation performed slightly better than the classifier using gender frequencies.

\cite{bergsma2005automatic} received with a similar approach using corpus gender frequencies an accuracy of 73.3 \%. In order to make both approaches as comparable as possible, another classifier lowering the threshold if no accepted antecedent was found in a first step was implemented. The approach of \cite{bergsma2005automatic} slightly increased the accuracy by 1.4 \% (to a total of 63.3 \%). 

Another legitimate evaluation measure is that not the most recent antecedent exceeding the threshold is accepted, but the candidate exceeding the threshold the most. This approach was also implemented and achieved an accuracy of 51.5 \% and an F-measure of 68 \%. The feature set including corpus mined gender frequencies was used for this implementation.

In order to make the results more comprehensible, an error analysis on six randomly chosen documents was done.

\section{Error Analysis}
\begin{table}[h]
\centering
  \caption{Types and frequencies of errors}
\begin{tabular}{|l|l|l|}
	\hline
	Type & Frequency & Percentage \\ \hline
	\hline
	Another previous instance detected & 15 & 30.6 \% \\ \hline
	Wrong antecedent & 10 & 20.4 \% \\ \hline
	Gender mismatch & 8 & 16.3 \% \\ \hline
	No accepted antecedent & 7 & 14.3 \% \\ \hline
	Wrong bound & 6 & 12.2 \% \\ \hline
	Other & 3 & 8.1 \% \\ \hline
	\end{tabular}

     \label{table:errorFreq}
\end{table}

The errors made by the anaphora resolution system can be divided into six different categories (Table \ref{table:errorFreq}).

The most frequent error was that an instance of the requested anaphora was found, but not the nearest. For instance, in Figure \ref{figure:prevInstError} \textit{Kirk} was detected while \textit{minister of Aberfoyle parish} was the correct antecedent. However, both refer to the same real-world entity. Since a classifier detecting another noun phrase representing the same real-world entity is only partially wrong, the impact on the performance score of this error might be relevant as well. Therefore, a seperate SVM classifier with corpus gender frequencies was built which also accepts previous real-world entities as the correct antecedent. This information was derived through coreference chains (pictured in Figure \ref{figure:nlppipeline}). A 10-fold crossvalidation yielded an accuracy increase of 5.1 \%.

\begin{figure}[h]
\centering
	\fbox{\scalebox{0.8}{
\xytext{
  \xybarnode{It was after this, while} &&
  \xybarnode{\fbox{Kirk}} &&
 \xybarnode{was} &&
  \xybarnode{\fbox{minister of Aberfoyle parish}} &&
  \xybarnode{, that} &&
  \xybarnode{\fbox{he}}
\xybarconnect[][--](D,D){-4} 
\xybarconnect(U,U){-8} &&
  \xybarnode{died in unusual circumstances .} 
}
}
}
\caption{Previous instance error example}
	\label{figure:prevInstError}
\end{figure}

In 10 of 49 cases, a wrong antecedent that does not explicitly disagree in gender was detected. Figure \ref{figure:wrongAnteError} shows an example of this case: \textit{they} referred to \textit{the French} but \textit{The Swedes} was detected instead. 

\begin{figure}[h]
\centering
	\fbox{\scalebox{0.8}{
\xytext{
  \xybarnode{\fbox{The Swedes}} &&
 \xybarnode{were also allied to} &&
  \xybarnode{\fbox{the French}} &&
  \xybarnode{, but} &&
  \xybarnode{\fbox{they}}
\xybarconnect[][--](D,D){-4} 
\xybarconnect(U,U){-8} &&
  \xybarnode{played no part in the battle .} &&
}
}
}
\caption{Wrong antecedent error example}
	\label{figure:wrongAnteError}
\end{figure}

Figure \ref{figure:wrongGenderError} shows a case in which the system clearly identified an noun phrase of the wrong gender. The corpus mined gender frequencies indicating a $\mu$ of approximately 90.5 \% for \textit{Sarah} to be female while its masculine $\mu$ was only 2.4 \%.

\begin{figure}[h]
\centering
	\fbox{\scalebox{0.8}{
\xytext{
\xybarnode{...} &&
  \xybarnode{\fbox{Sarah}} &&
 \xybarnode{continued to run it in partnership with} &&
  \xybarnode{\fbox{local brewer Samuel Mason}} &&
  \xybarnode{. Upon} &&
  \xybarnode{\fbox{his}} 
\xybarconnect[][--](D,D){-4} 
\xybarconnect(U,U){-8} &&
  \xybarnode{retirement ...} &&
}
}
}
\caption{Gender mismatch example}
	\label{figure:wrongGenderError}
\end{figure}

No accepted antecedent means that the classifier has searched backwards through the current and previous sentence and none of the examined noun phrases has been detected as the antecedent. In four of those cases the antecedent was more than one sentence distant and therefore no correct antecedent could be found by the algorithm. 

A wrong bound means that the correct antecedent might be detected, but not exactly as the annotation scheme expected:
\begin{addmargin}[25pt]{25pt}
Sarah Eldridge's son-in-law John Tizard inherited her share of the business, and when he died in 1871 the Popes assumed full control.
\end{addmargin}
In the example above, \textit{Sarah Eldridge's son-in-law John Tizard} was labelled as the correct antecedent for the pronoun \textit{he} while the system detected \textit{Sarah Eldridge's son-in-law}. One could argue that the system nonetheless detected the correct phrase. However, in some cases another span might change the complete meaning of the phrase. For instance, if the detected noun phrase is \textit{Gonzales 's musical style} and the correct antecedent is \textit{Gonzales's}.

The least frequent category of errors includes all remaining error types caused by parsing errors, other noise, or unclear bindings. For instance, \textit{their} in Figure \ref{figure:otherError} could either refer to \textit{Their commanders} or \textit{the Swedish troops}

\begin{figure}[h]
\centering
	\fbox{\scalebox{0.8}{
\xytext{
  \xybarnode{\fbox{Their commanders}} &&
 \xybarnode{waited for some time for} &&
  \xybarnode{\fbox{the Swedish troops}} &&
  \xybarnode{to appear on the open fields to} &&
  \xybarnode{\fbox{their}} 
\xybarconnect[2][--](D,D){-8} 
\xybarconnect(U,U){-4} &&
  \xybarnode{front.} &&
}
}
}
\caption{Ambiguity example}
	\label{figure:otherError}
\end{figure}



\chapter{Conclusion}
\label{sec:Conclusion}

\section{Summary}
What was done? What was learnt?

\section{Outlook}
What can/has to be/may be done in future research? Impact on other branches of science? society?



% Anhang
\cleardoublepage
%\pagenumbering{roman}
\part*{Appendix}
\begin{appendix}
\chapter{Appendix}

\begin{table}[h]
	\centering
	     \caption{Part-of-Speech Tagset}
     \label{table:AllPOSTags}
    \begin{tabular}{| l | l | l |}
    \hline
    Number & Tag & Description \\ \hline
	\hline
1. & 	CC &	Coordinating conjunction \\ \hline
2. & 	CD &	Cardinal number \\ \hline
3. & 	DT  & Determiner \\ \hline
4. & 	EX  &	Existential there \\ \hline
5. & 	FW &	Foreign word \\ \hline
6. & 	IN  &	Preposition or subordinating conjunction \\ \hline
7. & 	JJ &	Adjective \\ \hline
8. & 	JJR &	Adjective, comparative \\ \hline
9. & 	JJS &	Adjective, superlative \\ \hline
10. & 	LS &	List item marker \\ \hline
11. & 	MD &	Modal \\ \hline
12. & 	NN &	Noun, singular or mass \\ \hline
13. & 	NNS &	Noun, plural \\ \hline
14. & 	NNP& 	Proper noun, singular \\ \hline
15. & 	NNPS &	Proper noun, plural \\ \hline
16. & 	PDT &	Predeterminer \\ \hline
17. & 	POS &	Possessive ending \\ \hline
18. & 	PRP &	Personal pronoun \\ \hline
19. & 	PRP&	Possessive pronoun \\ \hline
20. & 	RB &	Adverb \\ \hline
21. & 	RBR &	Adverb, comparative \\ \hline
22. & 	RBS &	Adverb, superlative \\ \hline
23. & 	RP &	Particle \\ \hline
24. & 	SYM &	Symbol \\ \hline
25. & 	TO &	to \\ \hline
26. & 	UH &	Interjection \\ \hline
27. & 	VB &	Verb, base form \\ \hline
28. & 	VBD &	Verb, past tense \\ \hline
29. & 	VBG &	Verb, gerund or present participle \\ \hline
30. & 	VBN &	Verb, past participle \\ \hline
31. & 	VBP &	Verb, non-3rd person singular present \\ \hline
32. & 	VBZ &	Verb, 3rd person singular present \\ \hline
33. & 	WDT & Wh-determiner \\ \hline
34. & 	WP &	Wh-pronoun  \\ \hline
    \end{tabular}

\end{table}

\begin{table}[h]
	\centering
	      \caption{Constituency Tagset}
     \label{table:AllConstituencyTags}
    \begin{tabular}{| l | l | l |}
    \hline
Number & Tag & Description \\ \hline
	\hline
1. & 	ADJP & Adjective Phrase \\ \hline
2. & 	ADVP & Adverb Phrase \\ \hline
3. & 	CONJP  &  Conjunction Phrase \\ \hline
4. & 	FRAG  & Fragment \\ \hline
5. & 	INTJ & Interjection \\ \hline
6. & 	LST  &	 List marker \\ \hline
7. & 	NAC & Not a Constituent \\ \hline
8. & 	NP &	Noun Phrase \\ \hline
9. & 	NX &	Used within certain complex NPs to mark the head of the NP \\ \hline
10. & 	PP &  Prepositional Phrase \\ \hline
11. & 	PRN &  Parenthetical \\ \hline
12. & 	PRT &	Particle \\ \hline
13. & 	QP & Quantifier Phrase\\ \hline
14. & 	RRC& 	 Reduced Relative Clause\\ \hline
15. & 	UCP & Unlike Coordinated Phrase \\ \hline
16. & 	VP & Verb Phrase \\ \hline
17. & 	WHADJP & Wh-adjective Phrase \\ \hline
18. & 	WHAVP & Wh-adverb Phrase \\ \hline
19. & 	WHNP & Wh-noun Phrase \\ \hline
20. & 	WHPP &	Wh-prepositional Phrase \\ \hline
20. & 	X & Unknown, uncertain, or unbracketable\\ \hline
 \hline
21. & 	S &  Simple Declarative Clause \\ \hline
22. & SBAR &   Clause introduced by a subordinating conjunction \\ \hline
23. & 	SBARQ &  Direct question introduced by a wh-word or a wh-phrase \\ \hline
24. & 	SINV &   Inverted declarative sentence \\ \hline
25. & SQ & Inverted Yes/No Question, or Main Clause of a Wh-question\\ \hline

    \end{tabular}

\end{table}



\begin{table}[h]
	\centering
	      \caption{Dependency Tagset}
     \label{table:AllDependencyTags}
    \begin{tabular}{| l | l | l |}
    \hline
    Number & Tag & Description \\ \hline
	\hline
1. & 	acl  & clausal modifier of noun (adjectival clause) \\ \hline
2. &    advcl  & adverbial clause modifier\\ \hline
3. & 	advmod  & adverbial modifier \\ \hline
4. &    amod  & adjectival modifier \\ \hline
5. & 	appos  & appositional modifier \\ \hline
6. & 	aux  & auxiliary \\ \hline
7. &    auxpass  & passive auxiliary \\ \hline
8. & 	case  & case marking \\ \hline
9. & 	cc  & coordinating conjunction \\ \hline
10. &  ccomp  & clausal complement \\ \hline
11. &  compound  & compound \\ \hline
12. & 	conj  & conjunct \\ \hline
13. & 	cop  & copula \\ \hline
14. &  csubj  & clausal subject \\ \hline
15. &  csubjpass  & clausal passive subject \\ \hline
16. &  dep  & unspecified dependency \\ \hline
17. & det  & determiner \\ \hline
18. & discourse  & discourse element \\ \hline
19. &  dislocated  & dislocated elements \\ \hline
20. & 	dobj  & direct object \\ \hline
21. &  expl  & expletive \\ \hline
22. &  foreign  & foreign words \\ \hline
23. &  goeswith  & goes with \\ \hline
24. &  iobj  & indirect object \\ \hline
25. &  list  & list \\ \hline
26. &  mark  & marker \\ \hline
27. &  mwe  & multi-word expression \\ \hline
28. &  name  & name \\ \hline
29. & 	neg  & negation modifier \\ \hline
30. & 	nmod  & nominal modifier \\ \hline
31. & 	nsubj  & nominal subject \\ \hline
32. & 	nsubjpass  & passive nominal subject \\ \hline
33. & nummod  & numeric modifier \\ \hline
34. & 	parataxis  & parataxis  \\ \hline
35. & 	punct  & punctuation  \\ \hline
36. &  remnant  & remnant in ellipsis  \\ \hline
37. &  root  & root  \\ \hline
38. & 	vocative  & vocative  \\ \hline
39. & 	xcomp  & open clausal complement  \\ \hline
    \end{tabular}

\end{table}
\end{appendix}

% Verzeichnisse
\cleardoublepage
\part*{References}
\listoffigures
\listoftables
%\lstlistoflistings
\bibliography{literature}
\bibliographystyle{apacite}




\end{document}
