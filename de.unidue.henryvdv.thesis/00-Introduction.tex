\chapter{Introduction}
\label{sec:Introduction}

Anaphora resolution describes the procedure of determining whether two expressions - commonly noun phrases -  refer to each other, meaning that they are instances of the same entity. The word to be resolved is termed anaphora, while its predecessor is called the antecedent. It differs from coreference resolution by only resolving words, which can only be interpreted through its antecedent (Recasens et al., 2007) (1), while all corefering expressions are considered in coreference resolution (2).

(1) [Aberfoyle] describes [itself] as [The Gateway to [the Trossachs]]. \\
(resolve "itself" to "Aberfoyle")

(2) As late as 1790, all the residents in the parish of [Aberfoyle] spoke [Scottish Gaelic]. From 1882 [the village] was served by [Aberfoyle railway station].\\
(resolve "the village"  to "Aberfoyle")

Resolving noun phrases is a growing task in Natural Language Processing (NLP) and increased its relevance in the last decades, that it has even developed into a standalone subtask in the DARPA Message Understanding Conference in 1995 (MUC-6 1995). The International Workshop on Semantic Evaluation (SemEval) ran a coreference resolution task on multiple languages (Recasens et al., 2010), emphasizing the importance of coreference resolution systems. 
There are several important applications of coreference and anaphora resolution, such as Information Extraction (IE) (McCarthy and Lehnert, 1995), Question Answering (QA) (Morton, 2000), and Summarization (Steinberger et al., 2007).\\ 
Information Extraction has set itself the objective of summarizing relevant information from documents. Anaphora resolution is needed, because the sought entity is often referenced through different words (for instance personal pronouns). McCarthy and Lehnert described it as a classification problem: "given two references, do they refer to the same object or different objects."\\
The Question Answering task described by Morton has the goal to find a 250 byte string excerpt out of a number of documents as the answer to a query. Annotated coreference chains were used to link all instances of the same entity in a document. Occurrences in an other sentence are given a lower weight for prediction. The use of annotated coreference chains improved the prediction slightly.\\
Steinberger et al. figured out that the additional use of anaphoric information improved their performance score over solely Latent Semantic Analysis (LSA) summarization.

A lot of different information need to be used since selecting a possible antecedent is a decision under high ambiguity. The decisive factor for determination might be for instance cases gender-, or grammatical number agreement . Sometimes there is no decisive factor at all. Examples for gender agreement are shown in (3) and (4), for number agreement in (5) and (6).

(3) John and Jill had a date, but he didn't come. (resolve "he" to "John").

(4) John and Jill had a date, but she didn't come. (resolve "she" to "Jill").

(5) John loves his children. They are very nice. (resolve "they" to "his children").

(6) John loves his children. He is very nice. (resolve "he" to "John").

There are several approaches for coreference and anaphora resolution. 
In this work i will focus on pronominal anaphors, including reflexives, possessives, nominatives and predicates. 

Due to the difficulty of coreference and anaphora resolution, there are countless approaches, from rule-bases techniques to machine learning. 

In this work i will present a machine learning approach 
Pronoun resolution was contemplated in the SemEval coreference resolution task. 




establish your territory (say what the topic is about) and/or niche (show why there needs to be further research on your topic)
shortly introduce your research/what you will do in your thesis (make hypotheses; state the research questions)


Anaphora resolution is an important and growing subtask in Natural Language Processing. In the last decades, its relevance increased due to the amount of textual information in media. 